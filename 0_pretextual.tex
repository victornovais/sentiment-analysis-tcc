% --- -----------------------------------------------------------------
% --- Elementos usados na Capa e na Folha de Rosto.
% --- EXPRESSES ENTRE <> DEVERO SER COMPLETADAS COM A INFORMAO ESPECFICA DO TRABALHO
% --- E OS SMBOLOS <> DEVEM SER RETIRADOS 
% --- -----------------------------------------------------------------
\autor{<NOMES ALUNOS>} % deve ser escrito em maiusculo

\titulo{<TÍTULO>} % deve ser escrito em maiusculo

\instituicao{INSTITUTO INFNET} % deve ser escrito em maiusculo

\orientadora{YONA LOPES}% deve ser escrito em maiusculo - preencher com o nome do seu orientador

\coorientador{NOME} %preencher se houver.

\local{RIO DE JANEIRO} % deve ser escrito em maiusculo

\data{<ANO DA DEFESA>} % ano da defesa

\comentario{Trabalho de Conclusão de Curso apresentado ao Programa de Graduação em Engenharia da Computação do \mbox{Instituto Infnet} como parte dos requisitos necessários à obtenção do título de \mbox{Bacharel em Engenharia da Computação}.}%preencha com a sua area de concentracao


% --- -----------------------------------------------------------------
% --- Capa. (Capa externa, aquela com as letrinhas douradas)(Obrigatorio)
% --- ----------------------------------------------------------------
\capa

% --- -----------------------------------------------------------------
% --- Folha de rosto. (Obrigatorio)
% --- ----------------------------------------------------------------
\folhaderosto


\pagestyle{ruledheader}
\setcounter{page}{1}
\pagenumbering{roman}

% --- -----------------------------------------------------------------
% --- Ficha Catalográfica. (Obrigatorio)
% --- ----------------------------------------------------------------
%
%\cleardoublepage
%\thispagestyle{empty}
%\vspace*{130mm}
%
%\begin{flushright}
%
%\hspace{8em}
%\fbox{\begin{minipage}{10cm}

%....

%\end{minipage}}
%
%\end{flushright}
%\newpage

% --- -----------------------------------------------------------------
% --- Termo de aprovacao. (Obrigatorio)
% --- ----------------------------------------------------------------


\cleardoublepage
\thispagestyle{empty}

\vspace{-60mm}

\begin{center}
   {\large <NOMES ALUNOS>}\\
   \vspace{7mm}

  <TITULO DO TRABALHO>\\
%   \vspace{10mm}
 \vspace{8mm}  %diminui para ajustar a data que estava pulando para outra folha
\end{center}

\noindent
\begin{flushright}
\begin{minipage}[t]{8cm} 
%\begin{minipage}{\columnwidth}

Trabalho de Conclusão de Curso apresentado ao Programa de Graduação em Engenharia da Computação do \mbox{Instituto Infnet} como parte dos requisitos
necessários à obtenção do título de \mbox{Bacharel em Engenharia da Computação}

\end{minipage}
\end{flushright}
\vspace{ 5mm}  %original 1cm
\noindent
Aprovada em XX dezembro de XXXX. \\
\begin{flushright}
  \parbox{15cm}
  {
  \begin{center}
  BANCA EXAMINADORA \\
  \vspace{5mm}
  \rule{11cm}{.1mm} \\
	Profª. Yona Lopes, M.Sc. - Orientadora \\ 
	INFNET e Universidade Federal Fluminense (UFF) \\
  \vspace{5mm}
  \rule{11cm}{.1mm} \\
    Profª. XXXX, titulacao.  \\
    Universidade \\
  \vspace{5mm}
  \rule{11cm}{.1mm} \\
    Profº. xxx, TITULACAO \\
    Universidade\\
  \end{center}
  }
\end{flushright}
\begin{center}
  \vspace{4mm} % 
 Rio de Janeiro \\
  %\vspace{6mm}
  201X
\end{center}

% --- -----------------------------------------------------------------
% --- Dedicatoria.(Opcional)
% --- -----------------------------------------------------------------
\cleardoublepage
\thispagestyle{empty}
\vspace*{200mm}

\begin{flushright}
{\em 
À minha família.
%Dedicatória(s): À blá blá%Elemento opcional onde o autor presta homenagem ou dedica seu trabalho (ABNT, 2005).
}
\end{flushright}
\newpage


% --- -----------------------------------------------------------------
% --- Agradecimentos.(Opcional)
% --- -----------------------------------------------------------------
\pretextualchapter{Agradecimentos}
\hspace{5mm}
Agradeço, inicialmente, 

% --- -----------------------------------------------------------------
% --- Resumo em portugues.(Obrigatorio)
% --- -----------------------------------------------------------------
\begin{resumo}



{\hspace{-8mm} \bf{Palavras-chave}}: xxxxx.

%redefine todas siglas para não usado
\acresetall

\end{resumo}

% --- -----------------------------------------------------------------
% --- Resumo em lingua estrangeira.(Obrigatorio)
% --- -----------------------------------------------------------------
\begin{abstract}


{\hspace{-8mm} \bf{Palavras-chave}}: xxxxxxx.


%redefine todas siglas para não usado
\acresetall

\end{abstract}

% --- -----------------------------------------------------------------
% --- Lista de figuras.(Opcional)
% --- -----------------------------------------------------------------
\cleardoublepage
\listoffigures
\label{listadefiguras}
\acresetall

% --- -----------------------------------------------------------------
% --- Lista de tabelas.(Opcional)
% --- -----------------------------------------------------------------
\cleardoublepage
\label{listadetabelas}
\listoftables
\cleardoublepage
\acresetall

% --- -----------------------------------------------------------------
% --- Lista de abreviatura.(Opcional)
%Elemento opcional, que consiste na relação alfabética das abreviaturas e siglas utilizadas no texto, seguidas das %palavras ou expressões correspondentes grafadas por extenso. Recomenda-se a elaboração de lista própria para cada %tipo (ABNT, 2005).
% --- ----------------------------------------------------------------
\cleardoublepage
\pretextualchapter{Lista de Abreviaturas e Siglas} %%%Baixo Exemplo. Apagar e colocar os proprios em ordem alfabetica
\begin{acronym}[MCAA]
 \acro{API}{\textit{Application Programming Interface}}
 \acro{BRP}{\textit{Beacon Redundancy Protocol}}
% \acro{CoS}{\textit{Class of Service}}
 \acro{CID}{\textit{Configured IED Description}}
 \acro{CFI}{\textit{Canonical Format Indicator}}
 \acro{CRP}{\textit{Cross-network Redundancy Protocol}}
 \acro{DA}{\textit{Data Attributes}}
 \acro{DO}{\textit{Data Objects}}
 \acro{DRP}{\textit{Distributed Redundancy Protocol}}
 \acro{GOOSE}{\textit{Generic Object Oriented Substation Event}}
 \acro{GMRP} {\textit{GARP Multicast Registration Protocol}}
 \acro{HMI}{\textit{Human Machine Interface}} 
 \acro{HSR}{\textit{High-availability Seamless Redundancy}}
 \acro{ICD}{\textit{IED Capability Description}}
 \acro{IEC}{\textit{International Electrotechnical Commission}}
 \acro{IED}{\textit{Intelligent Electronic Device}}
 \acro{IGMP} {\textit{Internet Group Management Protocol}}
 \acro{IP}{\textit{Internet Protocol}}
 \acro{LAN}{\textit{Local Area Network}}
 \acro{LD} {\textit{Logical Device}}
 \acro{LN} {\textit{Logical Node}}
 \acro{LLDP}{\textit{Link Layer Discovery Protocol}}
 \acro{MAC}{\textit{Media Access Control}}
 \acro{MCAA} {\textit{Multicast Application Association}}
 \acro{MMS}{\textit{Manufacturing Message Specification}}
 \acro{MPLS}{\textit{Multiprotocol Label Switching}}
 \acro{MRP}{\textit{Media Redundancy Protocol}}
% \acro{MU}{\textit{Merging Unit}}
 \acro{NAT} {\textit{Network Address Translation}}
% \acro{NTP}{\textit{Network Time Protocol}}
 \acro{PCap}{\textit{Packet Capture}}
 \acro{PCP}{\textit{Priority Code Point}}
 \acro{PDIS}{\textit{DIStance protection}}
 \acro{PIM-SSM}{\textit{Protocol Independent Multicast - Source Specific Multicast}}
 \acro{PMU} {\textit{Phasor Measurement Unit}}
 \acro{PTOC}{\textit{Time OverCurrent}}
 \acro{PTRC}{\textit{TRip Conditioning}}
 \acro{PRP}{\textit{Parallel Redundancy Protocol}} 
 \acro{ONF}{\textit{Open Networking Foundation}}
 \acro{QoS} {\textit{Quality of Service}} 
 \acro{RSTP}{\textit{Rapid Spanning Tree Protocol}}
 \acro{RTU}{\textit{Remote Terminal Unit}}
 \acro{SAS}{Sistema de Automação de Subestações}
 \acro{SCADA}{\textit{Supervisory Control and Data Acquisition}}
 \acro{SCD}{\textit{Substation Configuration Description}}
 \acro{SCL}{\textit{Substation Configuration Language}}
 \acro{SCTP}{\textit{Stream Control Transmission Protocol}}
 \acro{SDN} {\textit{Software Defined Network}}
 \acro{SEP}{Sistema Elétrico de Potência}
 \acro{SMARTFlow}{SisteMa Autoconfigurável para Redes de Telecomunicações IEC 61850 com arcabouço OpenFlow}
 \acro{SMV}{\textit{Sampled Measured Values}}
 \acro{SPF}{\textit{Shortest Path First}}
 \acro{SSD}{\textit{System Specification Description}}
 \acro{SSL} {\textit{Secure Socket Layer}}
 \acro{STP}{\textit{Spanning Tree Protocol}}
 \acro{SV}{\textit{Sampled Values}}
 \acro{TC} {Transformador de Corrente}
 \acro{TCP}{\textit{Transmission Control Protocol}}
 \acro{TCAM}{\textit{Ternary Content Addressable Memory}}
 \acro{TCI}{\textit{Tag Control Information}}
 \acro{TLV}{\textit{Type Length Value}}
 \acro{TP} {Transformador de Potencial}
 \acro{TPAA} {\textit{Two Party Application Association}}
 \acro{TPID}{\textit{Tag Protocol Identifier}}
 \acro{VID} {\textit{VLAN Identifier}}
 \acro{VLAN}{\textit{Virtual Local Area Network}}
 \acro{XML}{\textit{eXtensible Markup Language}}
\end{acronym}

% --- -----------------------------------------------------------------
% --- Sumario.(Obrigatorio)
% --- -----------------------------------------------------------------
\pagestyle{ruledheader}
\tableofcontents
\acresetall


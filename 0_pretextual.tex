% --- -----------------------------------------------------------------
% --- Elementos usados na Capa e na Folha de Rosto.
% --- EXPRESSES ENTRE <> DEVERO SER COMPLETADAS COM A INFORMAO ESPECFICA DO TRABALHO
% --- E OS SMBOLOS <> DEVEM SER RETIRADOS 
% --- -----------------------------------------------------------------
\autor{NICOLAS DE SOUSA TEODOSIO E VICTOR HUGO NOVAIS RODRIGUES} % deve ser escrito em maiusculo

\titulo{ANÁLISE DE SENTIMENTO E MINERAÇÃO DE OPINIÕES APLICADO NO TWITTER} % deve ser escrito em maiusculo

\instituicao{INSTITUTO INFNET} % deve ser escrito em maiusculo

\orientador{CASSIUS FIGUEIREDO}% deve ser escrito em maiusculo - preencher com o nome do seu orientador

%\coorientador{NOME} %preencher se houver.

\local{RIO DE JANEIRO} % deve ser escrito em maiusculo

\data{2016} % ano da defesa

\comentario{Trabalho de Conclusão de Curso apresentado ao Programa de Graduação em Engenharia da Computação do \mbox{Instituto Infnet} como parte dos requisitos necessários à obtenção do título de \mbox{Bacharel em Engenharia da Computação}.}%preencha com a sua area de concentracao


% --- -----------------------------------------------------------------
% --- Capa. (Capa externa, aquela com as letrinhas douradas)(Obrigatorio)
% --- ----------------------------------------------------------------
\capa

% --- -----------------------------------------------------------------
% --- Folha de rosto. (Obrigatorio)
% --- ----------------------------------------------------------------
\folhaderosto


\pagestyle{ruledheader}
\setcounter{page}{1}
\pagenumbering{roman}

% --- -----------------------------------------------------------------
% --- Ficha Catalográfica. (Obrigatorio)
% --- ----------------------------------------------------------------
%
%\cleardoublepage
%\thispagestyle{empty}
%\vspace*{130mm}
%
%\begin{flushright}
%
%\hspace{8em}
%\fbox{\begin{minipage}{10cm}

%....

%\end{minipage}}
%
%\end{flushright}
%\newpage

% --- -----------------------------------------------------------------
% --- Termo de aprovacao. (Obrigatorio)
% --- ----------------------------------------------------------------


\cleardoublepage
\thispagestyle{empty}

\vspace{-60mm}

\begin{center}
   {\large NICOLAS DE SOUSA TEODOSIO E VICTOR HUGO NOVAIS RODRIGUES}\\
   \vspace{7mm}

  ANÁLISE DE SENTIMENTO E MINERAÇÃO DE DADOS APLICADO NO TWITTER\\
%   \vspace{10mm}
 \vspace{8mm}  %diminui para ajustar a data que estava pulando para outra folha
\end{center}

\noindent
\begin{flushright}
\begin{minipage}[t]{8cm} 
%\begin{minipage}{\columnwidth}

Trabalho de Conclusão de Curso apresentado ao Programa de Graduação em Engenharia da Computação do \mbox{Instituto Infnet} como parte dos requisitos
necessários à obtenção do título de \mbox{Bacharel em Engenharia da Computação}

\end{minipage}
\end{flushright}
\vspace{ 5mm}  %original 1cm
\noindent
Aprovada em XX agosto de 2016. \\
\begin{flushright}
  \parbox{15cm}
  {
  \begin{center}
  BANCA EXAMINADORA \\
  \vspace{5mm}
  \rule{11cm}{.1mm} \\
	Profº. Cassius Figueired, M.Sc. - Orientador \\ 
	Instituto INFNET \\
  \vspace{5mm}
  \rule{11cm}{.1mm} \\
    Profª. XXXX, titulacao.  \\
    Universidade \\
  \vspace{5mm}
  \rule{11cm}{.1mm} \\
    Profº. xxx, TITULACAO \\
    Universidade\\
  \end{center}
  }
\end{flushright}
\begin{center}
  \vspace{4mm} % 
 Rio de Janeiro \\
  %\vspace{6mm}
  2016
\end{center}

% --- -----------------------------------------------------------------
% --- Dedicatoria.(Opcional)
% --- -----------------------------------------------------------------
\cleardoublepage
\thispagestyle{empty}
\vspace*{200mm}

\begin{flushright}
{\em 
À minha família.
%Dedicatória(s): À blá blá%Elemento opcional onde o autor presta homenagem ou dedica seu trabalho (ABNT, 2005).
}
\end{flushright}
\newpage


% --- -----------------------------------------------------------------
% --- Agradecimentos.(Opcional)
% --- -----------------------------------------------------------------
\pretextualchapter{Agradecimentos}
\hspace{5mm}
Agradeço, inicialmente, 

% --- -----------------------------------------------------------------
% --- Resumo em portugues.(Obrigatorio)
% --- -----------------------------------------------------------------
\begin{resumo}

Atualmente a internet e micro blogs em geral têm se tornado uma ferramenta de comunicação poderosa entre usuários de Internet. Bilhões de pessoas compartilham informações e opiniões todos os dias, fazendo desse espaço um ótimo campo de pesquisas comercias, acadêmicas e sociológicas.  Como o fenômeno é relativamente recente – o Twitter foi criado apenas em 2006 – ainda existem poucas pesquisas destinadas ao tema.

Os principais desafios para aplicação dessa técnica estão relacionados a linguagens naturais sensíveis ao contexto que não trazem resultados satisfatórios quando utilizam-se modelos matemáticos muito simples, sendo necessário um grande investimento de tempo em aperfeiçoar os modelos matemáticos disponíveis e adaptá-los à solução em questão.

Outro desafio interessante é a aplicação de técnicas de mineração de opiniões no português, onde não existem muitos trabalhos relacionados e massas de treino disponíveis para consulta.

O objetivo deste trabalho é explorar o potencial existente em pesquisas de opinião que podem ser feitas através de análises nas comunicações feitas em língua portuguesa nas redes sociais todos os dias.

{\hspace{-8mm} \bf{Palavras-chave}}: Análise de sentimento, mídias sociais, twitter, mineração de opiniões, processamento de linguagem natural, linguagens sensíveis a contexto, naive bayes.

%redefine todas siglas para não usado
\acresetall

\end{resumo}

% --- -----------------------------------------------------------------
% --- Resumo em lingua estrangeira.(Obrigatorio)
% --- -----------------------------------------------------------------
\begin{abstract}


{\hspace{-8mm} \bf{Palavras-chave}}: xxxxxxx.


%redefine todas siglas para não usado
\acresetall

\end{abstract}

% --- -----------------------------------------------------------------
% --- Lista de figuras.(Opcional)
% --- -----------------------------------------------------------------
\cleardoublepage
\listoffigures
\label{listafiguras}
\acresetall

% --- -----------------------------------------------------------------
% --- Lista de tabelas.(Opcional)
% --- -----------------------------------------------------------------
\cleardoublepage
\label{listatabelas}
\listoftables
\cleardoublepage
\acresetall

% --- -----------------------------------------------------------------
% --- Lista de abreviatura.(Opcional)
%Elemento opcional, que consiste na relação alfabética das abreviaturas e siglas utilizadas no texto, seguidas das %palavras ou expressões correspondentes grafadas por extenso. Recomenda-se a elaboração de lista própria para cada %tipo (ABNT, 2005).
% --- ----------------------------------------------------------------
\cleardoublepage
\pretextualchapter{Lista de Abreviaturas e Siglas}
\begin{acronym}
	\acro{API}{Application Program Interface}
	\acro{PNL}{Processamento de Linguagem Natural}
	\acro{IA}{Inteligência Artificial}
	\acro{JSON}{\textit{Javascript Object Notation}}
	\acro{XML}{\textit{Extensible Markup Language}}
	\acro{FIFO}{\textit{First-In-First-Out}}
	
\end{acronym}

% --- -----------------------------------------------------------------
% --- Sumario.(Obrigatorio)
% --- -----------------------------------------------------------------
\pagestyle{ruledheader}
\tableofcontents
\acresetall


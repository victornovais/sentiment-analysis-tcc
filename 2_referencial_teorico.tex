\chapter{Referencial Teórico}\label{cap:referencial_teorico}

Normalmente o Capítulo~\ref{cap:XX} é parte do referencial teórico/estudo da arte/histórico... Aqui você precisa dar conhecimento para o leitor entender a sua proposta, deixando claro qual o problema. Pode ter mais de um capítulo no caso.


Continue usando citações como esta de exemplo~\cite{fangxing2010,gungor2011}.

Para itens estrangeiros como \textit{feedback}  não esqueça de colocar em itálico. Afirmações como essa: são os mesmos de 40 anos atrás~\cite{gungor2011}, precisam de referencia. 

\section{Twitter}\label{sec:twitter}

* Como começou
* Do que é feito , 140 caracteres
* Marcadores, hashtag
* Identificação de idioma
* Relevância na internet

O \textit{twitter} é conhecido como um \textit{microblog} fundado em março de 2006 por Jack Dorsey, Evan Williams e Biz Stone. Ele consiste em pequenos pedaços de informação de até 140 caracteres, que muitas vezes são utilizados pelos usuário para passar um sentimento com o intuito de expressas o seu \textit{status} naquele momento. O twitter faz uso de marcadores conhecidos como \textit{hashtag}
e os usuário fazem o uso desse marcador da seguinte forma, hashtag-palavra, com isso palavra se torna um marcador e o twitter começa a agregar qualquer usuário que faça a mesma ação. Esse uso massivo de marcadores e agregação foi criado os \textit{trending topics}, onde é mostrado o qual relevante aquele marcador está em determinado lugar, de escolha do usuário, como: Brasil, Mundo, Rio de Janeiro ou outra qualquer localidade.O twitter de acordo com \cite{arneromannkurrik2013} faz uso de \textit{machine learning} para identificar e classificar o idioma da mensagem escrita pelo usuário.


\section{API}\label{sec:api}
* O que é uma api
* Diferença de api x crawler
* Apis mais utilizadas

\section{Mineração de dados}\label{sec:mineracao_dados}
* O que é mineração de dados
* Como é feita
* Dificuldades
* Exemplos no mercado


\section{Processamento de linguagem natural}\label{sec:nlp}
* O que é processamento de linguagem natural
* O que é linguagem e linguagem natural
* Maiores dificuldades

\section{Classificação}\label{sec:classificacao}
* O que é classificar
* Contexto histórico




\section{Naive Bayes}\label{sec:naive_bayes}
* O que é o Naive Bayes
* Demonstração matemática do algoritmo
* Uso dele em analise de sentimento/classificação
\\ \emph{Naive Bayes} é um algoritmo probabilístico. Baseado no teorema de bayes. $$ P(A \mid B) = \frac{P(B \mid A) \, P(A)}{P(B)} $$ onde se infere qual é a probabilidade de um evento A dado um evento B. Porém nesse trabalho é utilizado o \emph{Naive Bayes} e sua diferença para o teorema de Bayes é assumir que a posição das palavras que aparecem no texto não importa, daí é acrescentado o \emph{naive}(ingênuo) ao teorema.
\\ Como visto em \cite{lucca2013implementaccao} o algoritmo computa qual a probabilidade de uma frase, denominada de documento pertencer a uma determinada classe(polaridade) \emph{P(c/d)}, a partir da probabilidade a \emph{priori} de \emph{P(c)} do documento pertencer a esta classe e da probabilidades condicionais de cada termo \emph{tk} ocorrer em um documento da mesma classe. O algoritmo tem como objetivo encontrar a melhor classe para um documento maximizando a probabilidade a\emph{posteriori} conforme a equação abaixo, onde $ n_{d} $ é o número de termos no documento \emph{d}. $$ C_{map}= argmax_{c \epsilon C}P(c|d)=argmax_{c \epsilon C}P(c)\prod 1sksn_{d}P(t_{k}/d) $$

\section{Análise de sentimento}\label{sec:analise_sentimento}
* O que é sentimento
* Como fazer a análise
* Como tirar insumo
* Exemplos 

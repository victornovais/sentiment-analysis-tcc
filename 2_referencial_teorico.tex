\chapter{Referencial Teórico}\label{cap:referencial_teorico}

Normalmente o Capítulo~\ref{cap:XX} é parte do referencial teórico/estudo da arte/histórico... Aqui você precisa dar conhecimento para o leitor entender a sua proposta, deixando claro qual o problema. Pode ter mais de um capítulo no caso.


Continue usando citações como esta de exemplo~\cite{fangxing2010,gungor2011}.

Para itens estrangeiros como \textit{feedback}  não esqueça de colocar em itálico. Afirmações como essa: são os mesmos de 40 anos atrás~\cite{gungor2011}, precisam de referencia. 

\section{API}\label{sec:api}


\section{Mineração de dados}\label{sec:mineracao_dados}


\section{Processamento de linguagem natural}\label{sec:nlp}


\section{Análise de sentimento}\label{sec:analise_sentimento}


\section{Naive Bayes}\label{sec:naive_bayes}


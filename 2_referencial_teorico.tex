\chapter{Referencial Teórico}\label{cap:referencial_teorico}

Normalmente o Capítulo~\ref{cap:XX} é parte do referencial teórico/estudo da arte/histórico... Aqui você precisa dar conhecimento para o leitor entender a sua proposta, deixando claro qual o problema. Pode ter mais de um capítulo no caso.


Continue usando citações como esta de exemplo~\cite{fangxing2010,gungor2011}.

Para itens estrangeiros como \textit{feedback}  não esqueça de colocar em itálico. Afirmações como essa: são os mesmos de 40 anos atrás~\cite{gungor2011}, precisam de referencia. 

\section{Twitter}\label{sec:twitter}

* Como começou
* Do que é feito , 140 caracteres
* Marcadores, hashtag
* Identificação de idioma
* Relevância na internet

O \textit{twitter} é conhecido como um \textit{microblog} fundado em março de 2006 por Jack Dorsey, Evan Williams e Biz Stone. Ele consiste em pequenos pedaços de informação de até 140 caracteres, e muitas vezes esses 140 caracteres são utilizados pelos usuário para passar um sentimento com o intuito de expressas o seu \textit{status} naquele momento. O twitter faz uso de marcadores conhecidos como \textit{hashtag}
e os usuário fazem o uso desse marcador da seguinte forma, hashtag-palavra, com isso palavra se torna um marcador e o twitter começa a agregar qualquer usuário que faça a mesma ação. Esse uso massivo de marcadores e agregação foi criado os \textit{trending topics}, onde é mostrado o qual relevante aquele marcador está em determinado lugar, de escolha do usuário, como: Brasil, Mundo, Rio de Janeiro ou outra qualquer localidade.O twitter de acordo com o \cite{arneromannkurrik2013} faz uso de \textit{machine learning} para identificar e classificar o idioma da mensagem escrita pelo usuário.


\section{API}\label{sec:api}
* O que é uma api
* Diferença de api x crawler
* Apis mais utilizadas

\section{Mineração de dados}\label{sec:mineracao_dados}
* O que é mineração de dados
* Como é feita
* Dificuldades
* Exemplos no mercado


\section{Processamento de linguagem natural}\label{sec:nlp}
* O que é processamento de linguagem natural
* O que é linguagem e linguagem natural
* Maiores dificuldades


\section{Análise de sentimento}\label{sec:analise_sentimento}
* O que é sentimento
* Como fazer a análise
* Como tirar insumo
* Exemplos 



\section{Naive Bayes}\label{sec:naive_bayes}
* O que é o Naive Bayes
* Demonstração matemática do algoritmo
* Uso dele em analise de sentimento/classificação
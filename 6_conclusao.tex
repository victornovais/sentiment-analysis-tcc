\chapter{Conclusão} \label{cap:conclusao}
Este trabalho identificou e abordou a importância que a mineração de opinião possui nos dias atuais, além das oportunidades que este novo campo de estudo abre com a sua descoberta: tanto na área acadêmica quanto para exploração comercial e estratégica e assim como propôs, desenvolveu uma solução capaz de extrair mensagens escritas em língua portuguesa de uma \textit{hashtag} do Twitter e classificá-las em positivas, neutras ou negativas, tornando possível fazer análises de mineração de opinião que são capazes de interpretar como um universo de usuários se sente em relação a temática estudada. Sem dúvida, o grande desafio foi mostrar que é viável aplicar uma solução de mineração de opinião em língua portuguesa.
Com base no que foi apresentado, a proposta cumpriu os seguintes objetivos levantados:

\begin{itemize}
	\item Utilização do Twitter como plataforma de dados;
	\item Construção um processo escalável capaz de extrair mensagens de uma \textit{hashtag} do Twitter;
	\item Utilização bases de palavras classificadas disponíveis na literatura sobre o tema;
	\item Aplicação de técnicas de normalização de texto visando um maior desempenho ao classificar;
	\item Implementação do algoritmo de \textit{Naive Bayes} para classificar as mensagens em positivas, neutras ou negativas;
	\item Conduzir a classificação em mensagens que foram escritas em língua portuguesa;
	\item Aplicação do conhecimento adquirido durante este trabalho em um estudo de caso utilizando a cerimônia do Oscars 2016;
	\item Demonstração dos cenários de teste e seus impactos no resultado final;
	\item Análise dos resultados obtidos atraves de gráficos que permitem interpretar os sentimentos dos usuários antes, durante e depois do evento, sob diversas perspectivas.
\end{itemize}

A análise dos resultados mostrou que por existir uma ansiedade muito grande vinculada aos instantes que antecedem o evento existe uma grande concentração de sentimentos positivos atrelados a este momento. Outro fato que ficou evidente é que os momentos mais populares do evento, como o anúncio das categorias mais aguardadas como: melhor ator, melhor atriz e melhor filme são as mais esperadas e com a confirmação da vitória dos favoritos, houve um pico de mensagens positivas por volta do horário do anúncio dos vencedores destas categorias. Outros dois momentos chamaram atenção pela carga de sentimentos negativos: o show de David Grohl em homenagem aos falecidos no mundo do cinema em 2015 onde a sensação de luto ficou evidente e a apresentação de Lady Gaga, onde o tema do filme que a música  faz parte da trilha sonora aborda estupros em faculdades americanas, tema que desperta bastante tristeza nas pessoas que sofreram este tipo de trauma ou que empatizam de alguma forma com as vítimas deste crime.


\section{Trabalhos Futuros}\label{sec:8_trabfut}

Como visto neste trabalho, o algoritmo utilizado para gerar os resultados das classificações afim de realizar a análise de sentimento foi o \textit{Naive Bayes}. Para trabalhos futuros pode-se utilizar outros algoritmos como: Árvore de Decisão, Regressão Logística, \textit{Maximum Entropy Model}. Podendo ser usados somente o próprio algoritmo ou realizando um trabalho comparativo entre eles, descobrindo qual algoritmo gera melhores resultados.

Outro método que pode ser utilizado é a aplicação de \textit{n-grams}, para averiguar o melhor desempenho do modelo. Com isso poderia-se realizar testes de acurácia, que não foi o enfoque deste trabalho. Pode-se também utilizar outras fontes de dados, não só o Twitter, como: Facebook, fórums, comentários de sites ,entre outros. Mais um fator relevante são as bases, que nesse trabalho foram utilizadas três bases genéricas e uma pequena base personalizada, então como trabalho futuro propõem-se utilizar apenas bases personalizadas e referentes ao domínio.

Durante as pesquisas para esse trabalho foi encontrado um \textit{toolkit}  para a técnica de \textit{word embedding}, conhecido como \textit{word2vec}, criado em 2013 por uma equipe do Google. Esse tipo de técnica se baseia em redes neurais, que difere do algoritmo probabilístico utilizado nesse trabalho. Sendo essa uma outro proposta pra trabalhos futuros.

Entre os testes realizados no capítulo \ref{cap:resultados} foi presenciado um fator instigante. A duração dos testes,  que podem ser encontrados no apêndice \ref{teste_tempo},  foi notado um crescimento anômalo, entretanto não foi discutido a causa de tal crescimento, então é proposto um estudo onde se discute tal anomalia. 

Por fim, pretende-se fazer o máximo para desenvolver o  processamento de linguagem natural para a língua portuguesa. Onde técnicas e métodos serão comparadas visando o melhor desempenho com uma análise mais profunda e a automação da inteligência artificial para o aproveitamento desse campo para o ser humano em suas tomadas de decisão. 

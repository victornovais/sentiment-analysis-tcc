\chapter{Conclusão} \label{cap:conclusao}

Um paragrágo relembrando a importância do cenário

Esse trabalho identificou e abordou alguns desses problemas, assim como propôs, desenvolveu e avaliou um serviço de gerenciamento eXXXXX
Relembrar o que o trabalho fez.


A proposta, XXX,  se destacou pelo XXXX que apresentou quando comparada XXXX. 

A proposta atingiu os seguintes objetivos, exemplo:
\begin{itemize}
\item permitiu que sejam usados IEDs mais simples pois a solução não precisa ser implementada nesses dispositivos;
\item reduziu o tempo de convergência dos algoritmos, o atraso na entrega de dados e o tráfego na rede;
\item atendeu aos requisitos da Norma IEC 61850;
\item implementou e testou um encaminhamento \textit{multicast} independente de camadas e transparente aos dispositivos finais;
\item permitiu uma configuração da rede facilitada;
\item usou o arquivo SCD da norma para autoconfiguração da rede de Telecomunicações;
\item tornou a rede menos sujeita à erros por ser automático;
\item permitiu o uso mais inteligente de recuperação de falhas;
\item permitiu o alcance de tempos de resposta menores por possuir uma característica proativa.
\end{itemize}

Os experimentos e as análises realizadas mostraramXXXXXX

Falar de todos os resultados encontrados de forma sumarizada, máximo de uma folha.

Os testes mostraram, também, que 


Outro ganho relacionado ao uso da técnica....

A análise realizada mostra que ...

\section{Trabalhos Futuros}\label{sec:8_trabfut}

Como visto neste trabalho, o algoritmo utilizado para gerar os resultados das classificações afim de realizar a análise de sentimento foi o \textit{Naive Bayes},. Para trabalhos futuros pode-se utilizar outros algoritmos como: Árvore de Decisão, Regressão Logística, \textit{Maximum Entropy Model}. Podendo ser usados somente o próprio algoritmo ou realizando um trabalho comparativo entre eles, descobrindo qual algoritmo gera melhores resultados.

Outro método que pode ser utilizado é a aplicação de \textit{n-grams}, para averiguar o melhor desempenho do modelo. Com isso poderia-se realizar testes de acurácia, que não foi o enfoque deste trabalho. Pode-se também utilizar outras fontes de dados, não só o Twitter, como: Facebook, fórums, comentários de sites ,entre outros. Mais um fator relevante são as bases, que nesse trabalho não tinham relação com o domínio estudado, pois eram genéricas, então pode-se realizar o mesmo trabalho com bases feitas específicas para o domínio.

Durante as pesquisas para esse trabalho foi encontrado um \textit{toolkit}  para a técnica de \textit{word embedding}, conhecido como \textit{word2vec}, criado em 2013 por uma equipe do Google. Esse tipo de técnica se baseia em redes neurais, que difere do algoritmo probabilístico utilizado nesse trabalho. Sendo essa uma outro proposta pra trabalhos futuros.

Por fim, pretende-se fazer o máximo para desenvolver o  processamento de linguagem natural para a língua portuguesa. Onde técnicas e métodos serão comparadas visando o melhor desempenho com uma análise mais profunda e a automação da inteligência artificial para o aproveitamento desse campo para o ser humano em suas tomadas de decisão. 
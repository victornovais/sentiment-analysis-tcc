\chapter{Proposta} \label{cap:proposta}

Definição da sua proposta.

Se for apresentar os algoritmos
use por exemplo:

\section{Trabalhos relacionados}\label{sec:trabalhos_relacionados}

\section{Implementação}\label{sec:implementacao}


\subsection{Crawler}


\subsection{Classificação}


\subsubsection{Algoritmo}


\subsubsection{Construção da base de dados}
A construção da base de dados foi feita com o intuito de melhor expressar um sentimento de uma palavra ou texto, para a utilização do algoritmo. Para isso a base foi dividida em dois arquivos, positivos e negativos. Além dessa divisão foi utilizada outas bases criadas como: Re-li(referencia), SentiLex-PT(referencia), base da puc(referencia), emoticons( referencia). Todas usando a língua portuguesa ou um linguajar universal, no caso dos emoticons e já estarem polarizadas. Essas bases têm em comum é serem feitas apenas de palavras, então ficou-se a dúvida de como a classificação funcionaria posteriormente quando aplicadas a um texto que as palavras podem não estar no mesmo contexto. Ex: "O flamengo jogou muito mal, mas fico feliz pela vitória", onde tem a palavra mal que já dá um tom negativo a frase , porém ao terminar de ler a frase encontrasse as palavras feliz e vitória que tem um contexto positivo.
Com essas bases já citadas foi compreendida a necessidade de uma base mais específica para o linguajar utilizado na internet, constituído de  
gírias, abreviação e até erros de português, para isso foi criada uma base utilizando dados pegos do twitter a partir da marcação hashtagoscar2016.


\subsubsection{Massa de treino}


\subsubsection{Massa de teste}


\subsection{Plataforma de análise}

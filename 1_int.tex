\chapter{Introdução} \label{cap:1}

Através do fenômeno da popularização da Internet vivemos hoje um período conhecido como "Era da conhecimento" \cite{lastres1999informaccao}.
Nesse contexto, redes sociais conhecidas, como Facebook e Twitter se tornaram bastante populares por permitirem a seus usuários acesso à um ambiente onde todos possuem voz e vez para se expressar e por consequência, para se informar sobre tudo que acontece no mundo.
Através de \ac{API} disponibilizadas por essas redes sociais, possuímos fácil acesso a um grande volume de opiniões catalogadas - através de \textit{hashtags} - que podem ser utilizadas em pesquisas de opinião sobre um tema ou assunto específico. Tal cenário apresenta-se como uma grande oportunidade de pesquisa em áreas acadêmicas, sociais e comerciais.
Porém, quando o objeto de estudo é a língua portuguesa, nota-se que a mesma carece de trabalhos e implementações na área de mineração de opiniões e análise de sentimento (REFERÊNCIA). Alguns motivos explicam essa carência: poucos investimentos na área de ciência e engenharia da computação em nosso país e a grande dificuldade que a língua portuguesa apresenta ao ser interpretada através de processamento de linguagem natural. \cite{santos2000projecto}

Com a crescente popularidade de blogs e redes sociais, as áreas de mineração de opinião e análise de sentimento se tornaram objeto de estudo de pesquisadores. Uma abordagem ampla sobre o assunto foi apresentada em Pang e Lee \cite{pang2008opinion}. Em seu trabalho, os autores descrevem diversas técnicas e abordagens aplicáveis em sistemas orientados à informação. Entre as diversas aplicações sugeridas, destacam-se abordagens que visam substituir sites especializados em resenhas e recomendações, propondo que sistemas possam buscar opiniões de usuários de forma proativa ao invés de esperar que o mesmo exponha seu parecer através da solicitação do preenchimento de um formulário de pesquisa, resenha ou comentário. Tal abordagem pode ser aplicada para pesquisas de opinião sobre produtos, pessoas e serviços.

Em Gomes \cite{gomes2013text} a mineração de texto é aplicada em busca de notícias sobre economia em Portugal. O trabalho concentra-se em monitorar sites relevantes que abordam notícias sobre a economia do país para representar o sentimento expresso no texto, através dos títulos das reportagens.

Em Pak e Paroubek \cite{pak2010twitter} o Twitter é utilizado como fonte dados para análises de sentimento. O idioma de estudo escolhido foi o inglês, mas grande parte das técnicas apresentadas podem ser aplicadas em outras línguas, visto que a coleta de dados e os algoritmos de classificação continuam inalteradas caso o objeto de estudo seja outro idioma.

Alguns trabalhos utilizam o português como objeto de estudo, como por exemplo Tortella e Coelho \cite{tortellaanalise}. Outros, se propõem a estudar um evento ou acontecimento finito, como por exemplo as eleições presidenciais no Brasil no ano de 2010 \cite{rodrigues2012characterizing}, os protestos populares contra a corrupção ocorridos em 2013 \cite{franca2014analise} e a Copa do Mundo da FIFA Brasil 2014 \cite{carvalho2014mineraccao}. Nesses casos, \textit{tweets} postados por usuários contendo \textit{hashtags} referentes ao evento a ser estudado são monitorados e salvos numa base de dados ao longo do evento. Após o fim do mesmo, os dados são classificados utilizando um algoritmo previamente treinado e os resultados são analisados a fim de determinar a relevância, impacto e opiniões geral de acordo com a opinião dos usuários.

Neste trabalho, será aplicado o processo de mineração de opinião e análise de sentimento de forma semelhante a Pak e Paroubek \cite{pak2010twitter}, porém utilizando português como idioma de estudo. O objetivo é demonstrar como a mineração de opinião e análise de sentimento podem ser abordadas de forma abrangente, com a aplicação de técnicas generalistas. Além disso, é proposto um estudo de caso sobre a cerimônia do Oscar no ano de 2016 para também mostrar como um evento específico pode ser estudado, aplicando técnicas mais específicas que tem como objetivo tornar a classificação mais especializada e precisa, similar ao que foi feito em \cite{rodrigues2012characterizing} \cite{franca2014analise} \cite{carvalho2014mineraccao}.


\section{Motivação e Objetivos}\label{sec:1_inicio}
%A motivação deste trabalho é explorar o potencial contido no conteúdo digital gerado todos os dias em redes sociais por %usuários brasileiros. Tais dados possuem informações valiosas que podem ser explorados de inúmeras maneiras



\section{Principais contribuições}\label{sec:1_principais_contribuicoes}

\section{Recursos utilizados}\label{sec:1_recursos_utilizados}

\section{Organização do trabalho}\label{sec:1_org}

Este trabalho está estruturado em 5 capítulos da seguinte forma: no Capítulo~\ref{cap:referencial_teorico}, para embasamento teórico, são apresentados os conceitos de (CONTINUA). Em seguida, no Capítulo~\ref{cap:proposta} , é feita uma análise sobre os principais trabalhos relacionados ao uso dos ... . No Capítulo~\ref{cap:referencial_teorico}, os conceitos do arcabouço utilizado ... , são descritos. Nesse capítulo são mostrados os motivos para a escolha desse arcabouço, .... A proposta XXX é apresentada no Capitulo~\ref{cap:proposta}, onde a arquitetura da proposta é detalhada, assim como seus componentes e algoritmos. Em seguida, o Capítulo~\ref{cap:resultados} apresenta as ferramentas utilizadas para implementação da proposta, o ambiente implementação, a descrição dos experimentos e os principais resultados obtidos com o XXX, assim como a análise dos valores encontrados. Por fim, o Capítulo~\ref{cap:conc} conclui este trabalho, ressaltando os objetivos alcançados com as propostas. As principais vantagens e desvantagens da proposta são discutidas, assim como alguns trabalhos futuros que podem ser desenvolvidos. 

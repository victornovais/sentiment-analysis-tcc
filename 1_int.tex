\chapter{Introdução} \label{cap:1}

Através do fenômeno da popularização da Internet vivemos hoje um período conhecido como "Era da conhecimento" \cite{lastres1999informaccao}.
Nesse contexto, redes sociais conhecidas, como Facebook e Twitter se tornaram bastante populares por permitirem a seus usuários acesso à um ambiente onde todos possuem voz e vez para se expressar e por consequência, para se informar sobre tudo que acontece no mundo.
Através de \ac{API} disponibilizadas por essas redes sociais, possuímos fácil acesso a um grande volume de opiniões catalogadas - através de \textit{hashtags} - que podem ser utilizadas em pesquisas de opinião sobre um tema ou assunto específico. Tal cenário apresenta-se como uma grande oportunidade de pesquisa em áreas acadêmicas, sociais e comerciais.
Porém, quando o objeto de estudo é a língua portuguesa, nota-se que a mesma carece de trabalhos e implementações na área de mineração de opiniões e análise de sentimento (REFERÊNCIA). Alguns motivos explicam essa carência: poucos investimentos na área de ciência e engenharia da computação em nosso país e a grande dificuldade que a língua portuguesa apresenta ao ser interpretada através de processamento de linguagem natural. \cite{santos2000projecto}


\section{Motivação e Objetivos}\label{sec:1_inicio}
%A motivação deste trabalho é explorar o potencial contido no conteúdo digital gerado todos os dias em redes sociais por %usuários brasileiros. Tais dados possuem informações valiosas que podem ser explorados de inúmeras maneiras



\section{Principais contribuições}\label{sec:1_principais_contribuicoes}

\section{Recursos utilizados}\label{sec:1_recursos_utilizados}

\section{Organização do trabalho}\label{sec:1_org}

Este trabalho está estruturado em 5 capítulos da seguinte forma: no Capítulo~\ref{cap:referencial_teorico}, para embasamento teórico, são apresentados os conceitos de (CONTINUA). Em seguida, no Capítulo~\ref{cap:proposta} , é feita uma análise sobre os principais trabalhos relacionados ao uso dos ... . No Capítulo~\ref{cap:referencial_teorico}, os conceitos do arcabouço utilizado ... , são descritos. Nesse capítulo são mostrados os motivos para a escolha desse arcabouço, .... A proposta XXX é apresentada no Capitulo~\ref{cap:proposta}, onde a arquitetura da proposta é detalhada, assim como seus componentes e algoritmos. Em seguida, o Capítulo~\ref{cap:resultados} apresenta as ferramentas utilizadas para implementação da proposta, o ambiente implementação, a descrição dos experimentos e os principais resultados obtidos com o XXX, assim como a análise dos valores encontrados. Por fim, o Capítulo~\ref{cap:conc} conclui este trabalho, ressaltando os objetivos alcançados com as propostas. As principais vantagens e desvantagens da proposta são discutidas, assim como alguns trabalhos futuros que podem ser desenvolvidos. 

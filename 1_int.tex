\chapter{Introdução} \label{cap:int}

Sua introdução precisa explicar de forma muito resumida o seu trabalho. Precisa motivar a leitura dele. Não esqueça de deixar claro o problema, os objetivos e sua motivação.


\section{Motivação e Objetivos}\label{sec:1_inicio}


\section{Principais contribuições}

\section{Recursos utilizados}

\section{Organização do trabalho}\label{sec:1_org}

Este trabalho está estruturado em 5 capítulos da seguinte forma: no Capítulo~\ref{cap:XX}, para embasamento teórico, são apresentados os conceitos de ... Neste capítulo, os conceitos relacionados a ..., dentre outros, são descritos. Em seguida, no Capítulo XX , é feita uma análise sobre os principais trabalhos relacionados ao uso dos ... . No Capítulo~\ref{cap:XX}, os conceitos do arcabouço utilizado ... , são descritos. Nesse capítulo são mostrados os motivos para a escolha desse arcabouço, .... A proposta XXX é apresentada no Capitulo~\ref{cap:prop}, onde a arquitetura da proposta é detalhada, assim como seus componentes e algoritmos. Em seguida, o Capítulo~\ref{cap:imp} apresenta as ferramentas utilizadas para implementação da proposta, o ambiente implementação, a descrição dos experimentos e os principais resultados obtidos com o XXX, assim como a análise dos valores encontrados. Por fim, o Capítulo~\ref{cap:conc} conclui este trabalho, ressaltando os objetivos alcançados com as propostas. As principais vantagens e desvantagens da proposta são discutidas, assim como alguns trabalhos futuros que podem ser desenvolvidos. 

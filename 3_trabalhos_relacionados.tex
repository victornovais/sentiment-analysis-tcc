\chapter{Trabalhos relacionados} \label{cap:trabalhos_relacionados}

Com a crescente popularidade de blogs e redes sociais, os campos de mineração de opinião e análise de sentimento se tornaram objeto de estudo de alguns pesquisadores. Uma abordagem ampla sobre o assunto foi apresentada em Pang e Lee \cite{pang2008opinion}. Em seu trabalho, os autores descrevem diversas técnicas e abordagens aplicáveis em sistemas orientados à informação. Entre as diversas aplicações sugeridas, destacam-se abordagens que visam substituir sites especializados em resenhas e recomendações, propondo que sistemas possam buscar opiniões de usuários de forma proativa ao invés de esperar que o mesmo exponha seu parecer através da solicitação do preenchimento de um formulário de pesquisa, resenha ou comentário. Tal abordagem pode ser aplicada para pesquisas de opinião sobre produtos, pessoas e serviços.

Em Gomes \cite{gomes2013text} a mineração de texto é aplicada em busca de notícias sobre economia de Portugal. O trabalho concentra-se em monitorar sites relevantes que abordam notícias sobre a economia do país para representar o sentimento expresso no texto, através dos títulos das reportagens.

Em Pak e Paroubek \cite{pak2010twitter} o Twitter é utilizado como fonte dados para análises de sentimento. O idioma de estudo escolhido foi o inglês, mas grande parte das técnicas apresentadas podem ser aplicadas em outras línguas, visto que a coleta de dados e os algoritmos de classificação continuam inalteradas caso o objeto de estudo seja outro idioma.

Alguns trabalhos utilizam o português brasileiro como objeto de estudo, como por exemplo Tortella e Coelho \cite{tortellaanalise}. Outros, se propõem a estudar um evento ou acontecimento finito, como por exemplo as eleições presidenciais no Brasil no ano de 2010 \cite{rodrigues2012characterizing}, os protestos populares contra a corrupção ocorridos em 2013 \cite{franca2014analise} e a Copa do Mundo da FIFA Brasil 2014 \cite{carvalho2014mineraccao}. Nesses casos, \textit{tweets} postados por usuários contendo \textit{hashtags} referentes ao evento a ser estudado são monitorados e salvos numa base de dados ao longo do evento. Após o fim do mesmo, os dados são classificados utilizando um algoritmo previamente treinado e os resultados são analisados afim de determinar a relevância, impacto e opiniões geral sobre de acordo com a opinião dos usuários.

Neste trabalho, será aplicado o processo de mineração de opinião e análise de sentimento de forma semelhante a Pak e Paroubek \cite{pak2010twitter}, porém utilizando português brasileiro como idioma de estudo. O objetivo é demonstrar como a mineração de opinião e análise de sentimento podem ser abordadas de forma abrangente, com a aplicação de técnicas generalistas. Além disso, é proposto um estudo de caso sobre a cerimônia do Oscar no ano de 2016 para também mostrar como um evento específico pode ser estudado, aplicando técnicas mais específicas que tem como objetivo tornar a classificação mais especializada e precisa, similar ao que foi feito em \cite{rodrigues2012characterizing} \cite{franca2014analise} \cite{carvalho2014mineraccao} .